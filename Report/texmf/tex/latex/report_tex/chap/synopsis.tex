
\chapter*{Synopsis}

In this report, we propose a safety-service that helps to reduce criminal offense count in public transport vehicles. A device called \emph{PanicButton device}* mounted in these vehicles is at the core of this service. The \emph{PanicButton device} is equipped with wireless modules like Global System for Mobile communications (GSM) for ubiquitous connectivity to the back-end servers and bluetooth low energy (BLE) for communicating with the users. To make the \emph{PanicButton device} user friendly, we restrict physical user interface to only a panic button. The \emph{PanicButton device} runs algorithm to detect any tampering and also provide smart-phone based validation. Audio-based trigger of this \emph{PanicButton device} provides the safety-service to disabled and immobilized victims. If scream is detected in audio received by the \emph{PanicButton device}, emergency is asserted. Support Vector Machine (SVM), a supervised machine learning method is used to  detect screams. F1 Score is used for benchmarking these SVMs, we achieved detection rate of 95\% for screams heard in laboratory environment. On receiving distress signal, police is notified and regularly updated with victim's location acquired using Global Positioning System (GPS), facilitating them to reach the victim.

*We refer to our product as `\emph{PanicButton device}', this notation is followed through out the report.